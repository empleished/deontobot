\pdfoutput=1

\documentclass{l4proj}

%
% put any packages here
%

\begin{document}
\title{Deontic Logic Prover}
\author{Leisha Hussien}
\date{2015/2016}
\maketitle

\begin{abstract}
A formalisation of deontic logic to represent the obligations, prohibitions and permissions within a system, which can then be checked for coherency and consistency to determine the validity of the set of rules.
\end{abstract}

\educationalconsent

\tableofcontents
%==============================================================================

\chapter{Introduction}
\pagenumbering{arabic}

\section{Deontic Logic}
Deontic logic is a logic of duty, which formalises conceptions of what should and should not be done. In this project, I will be primarily concerned with obligations, prohibitions and permissions. 
%Explain what it is, what it means to be obliged, prohibited, permitted, etc. Outline the variant of deontic logic in use, explain why things are being used/not used. Outline problems it faces and potential solutions. 

\section{Applications of Deontic Logic}
%Places it is already being used/could be used. 

\subsection{Existing Functionality}
%Things which already exist which kind of do what I'm doing - explain why they aren't sufficient and what I'm doing differently.

\chapter{Implementation}

\section{Lexical Specification}
%How all the pieces of the logic and situational features are represented. 

\section{Proof Strategy}
%How I make sure the rules actually make sense. 

% ** HOW TO INSERT FIGURES **
%\vspace{-7mm}
%\begin{figure}
%\centering
%\includegraphics[height=9.2cm,width=13.2cm]{uroboros.pdf}
%\vspace{-30mm}
%\caption{An alternative hierarchy of the algorithms.}1
%\label{uroborus}
%\end{figure}

%%%%%%%%%%%%%%%%
%              %
%  APPENDICES  %
%              %
%%%%%%%%%%%%%%%%
\begin{appendices}

\chapter{Running the Programs}
An example of running from the command line is as follows:
\begin{verbatim}
\end{verbatim}

\chapter{Test Data}
%include tests used

\end{appendices}

%%%%%%%%%%%%%%%%%%%%
%   BIBLIOGRAPHY   %
%%%%%%%%%%%%%%%%%%%%

\bibliographystyle{plain}
\bibliography{bib}

\end{document}
